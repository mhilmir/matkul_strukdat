% !TEX TS-program = pdflatex
% !TEX encoding = UTF-8 Unicode

% This is a simple template for a LaTeX document using the "article" class.
% See "book", "report", "letter" for other types of document.

\documentclass[11pt]{article} % use larger type; default would be 10pt
\usepackage{listings}  


\usepackage[utf8]{inputenc} % set input encoding (not needed with XeLaTeX)

%%% Examples of Article customizations
% These packages are optional, depending whether you want the features they provide.
% See the LaTeX Companion or other references for full information.

%%% PAGE DIMENSIONS
\usepackage{geometry} % to change the page dimensions
\geometry{a4paper} % or letterpaper (US) or a5paper or....
% \geometry{margin=2in} % for example, change the margins to 2 inches all round
% \geometry{landscape} % set up the page for landscape
%   read geometry.pdf for detailed page layout information

\usepackage{graphicx} % support the \includegraphics command and options

% \usepackage[parfill]{parskip} % Activate to begin paragraphs with an empty line rather than an indent

%%% PACKAGES
\usepackage{booktabs} % for much better looking tables
\usepackage{array} % for better arrays (eg matrices) in maths
\usepackage{paralist} % very flexible & customisable lists (eg. enumerate/itemize, etc.)
\usepackage{verbatim} % adds environment for commenting out blocks of text & for better verbatim
\usepackage{subfig} % make it possible to include more than one captioned figure/table in a single float
% These packages are all incorporated in the memoir class to one degree or another...

%%% HEADERS & FOOTERS
\usepackage{fancyhdr} % This should be set AFTER setting up the page geometry
\pagestyle{fancy} % options: empty , plain , fancy
\renewcommand{\headrulewidth}{0pt} % customise the layout...
\lhead{}\chead{}\rhead{}
\lfoot{}\cfoot{\thepage}\rfoot{}

%%% SECTION TITLE APPEARANCE
\usepackage{sectsty}
\allsectionsfont{\sffamily\mdseries\upshape} % (See the fntguide.pdf for font help)
% (This matches ConTeXt defaults)

%%% ToC (table of contents) APPEARANCE
\usepackage[nottoc,notlof,notlot]{tocbibind} % Put the bibliography in the ToC
\usepackage[titles,subfigure]{tocloft} % Alter the style of the Table of Contents
\renewcommand{\cftsecfont}{\rmfamily\mdseries\upshape}
\renewcommand{\cftsecpagefont}{\rmfamily\mdseries\upshape} % No bold!

%%% END Article customizations

%%% The "real" document content comes below...

\title{Laporan Tugas 1}
\author{Mochammad Hilmi Rusydiansyah\\5024211008}
%\date{} % Activate to display a given date or no date (if empty),
% otherwise the current date is printed 

\begin{document}
	\maketitle
		
	\section{Pendahuluan}
	Laporan ini dibuat untuk melengkapi tugas pertama dalam mata kuliah Struktur Data dan Analisa Algoritma Kelas (A). Dalam tugas 1 ini meliputi proposal pembuatan game, source code implementasi array abstract dalam bentuk oop, dan source code sistem game pesawat yang sudah dimodifikasi arah geraknya.
	\begin{figure}[h!]
		\centering
		\includegraphics[width=0.4\linewidth]{C++-Logo}
		\caption{Logo C++}
		\label{fig:C++-Logo}
	\end{figure}
	
	\section{Penjelasan Game}
	Dalam game ini, anda akan berperan sebagai penjaga kebun binatang. Suatu saat terjadi kebocoran zat kimia yang terletak di dekat kebun binatang. Kebocoran tersebut mengakibatkan hewan-hewan terkontaminasi zat kimia dan menjadi ganas, sehingga banyak dari hewan-hewan tersebut yang lepas dari kandang. Kondisi menjadi kacau, anda sebagai penjaga kebun binatang pun harus berusaha menghindari kejaran sambil memberi makan hewan-hewan yang ganas terseebut.
	
	\section{Cara Bermain}
	\begin{itemize}
		\item 
	\end{itemize}
	Mekanisme game yang digunakan adalah tembak-tembakan dengan perspektif kamera Top-down. Seakan-akan, pemain harus menembakkan peluru ke arah target yang terus bergerak mengejar pemain itu sendiri. Dengan kata lain, pemain harus melempar makanan ke arah hewan-hewan yang ganas, yang terus berusaha mengejar pemain itu sendiri. Apabila peluru makanan tepat mengenai hewan sasaran, maka hewan tersebut akan berhenti mengejar dan sebaliknya memakan peluru makanan tadi. Pemain harus berusaha mendapatkan skor tertinggi dengan bertahan selama mungkin.
	Pemain memiliki bar darah, apabila hewan berhasil mencapai pemain, darah pemain akan berkurang.

	\begin{lstlisting}
		#include <stdio.h>
		class ClArray {
			private:
			float data[100];
			int Size;
			int Length;
			public:
			ClArray()
			{
				Size = 100;
				Length = 0;
			}
			void Append(float Vi)
			{
				if (Length<Size)
				{
					Length++;
					data[Length-1]=Vi;
				}
				
			}
			void Display() 
			{
				int  i;
				for (i=0;i<Length;i++)
				{
					printf("%f\n",data[i]);
				}
				
			}
			
		};
		
		int main()
		{
			class ClArray d;
			d.Append(1);
			d.Display();
			return 0;
		}
		
	\end{lstlisting}
	
	Your text goes here.
	
	\subsection{A subsection}
	
	More text.
	
\end{document}
